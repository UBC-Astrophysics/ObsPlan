\documentclass[useAMS,usenatbib]{mn2e}
\usepackage{graphicx}
\usepackage{url}
%\usepackage{tikz}
\newcommand{\aap}{Astron. Astrophys.}
\newcommand{\aj}{Astron. J.}
\newcommand{\ao}{Appl. Opt.}
\newcommand{\apj}{Astrophys. J.}
\newcommand{\apjl}{Astrophys. J. Lett.}
\newcommand{\apjs}{Astrophys. J. Suppl.}
\newcommand{\mnras}{Mon. Not. Roy. Ast. Soc.}
\newcommand{\nat}{Nature}
\newcommand{\pasa}{Publ. Ast. Soc. Aust.}
\newcommand{\pasp}{Publ. Ast. Soc. Pac.}
\newcommand{\prl}{Phys. Rev. Lett.}
\newcommand{\msun}{\mathrm{M}_\odot}

\title[2MPZ and LIGO]{Using the 2-MASS Photometric Redshift Survey to optimize LIGO Follow-Up Observations}
\author[Antolini \& Heyl]{Elisa Antolini$^{1}$, Jeremy S. Heyl$\thanks{Email:
    heyl@phas.ubc.ca; Canada Research Chair}$^{2}$ \\
  $^{1}$Dipartimento di Fisica e Geologia, Universit\`a degli Studi di Perugia, I-06123 Perugia, Italia \\
  $^{2}$Department of Physics and Astronomy, University of British
  Columbia, 6224 Agricultural Road, Vancouver, BC V6T 1Z1, Canada\\
}
\begin{document}
\date{Accepted ---. Received ---; in original form ---}

\pagerange{\pageref{firstpage}--\pageref{lastpage}} \pubyear{2015}

\maketitle

\label{firstpage}

\begin{abstract}
The initial source localizations from LIGO-Virgo will cover large
areas of sky. Because astrophysical sources of gravitational radiation
are likely to reside in galaxies, it would make sense to search first
in regions where the LIGO-Virgo probability is large and where the
density of galaxies is large as well.  Under the Bayesian prior
assumption that the probability of a gravitational-wave event from a
given region of space is proportional to the density of galaxies
within the probed volume, one can calculate an improved localization
of the position of the source simply by multiplying the LIGO-Virgo
skymap by the density of galaxies in the range of redshifts.  We
propose using the 2-MASS Photometric Redshift Galaxy Catalogue for
this purpose and demonstrate that using it can dramtically reduce the
search region for electromagnetic counterparts.
\end{abstract}

\section{Introduction}

In the near future LIGO and Virgo should begin to detect graviataional
wave events from the local Universe.  During these initial years of
gravitational astronomy, the localization of the candidate events on
the sky will be poor with the ninety-percent confidence regions
covering hundreds or even thousands of square degrees.  Finding an
electromagnetic counterpart to these candidate graviational-wave
events will be crucial to understanding what produces them,
interpretation of the signal and to provide tests of general
relativity.  The ideas of how the the electromagnetic counterparts
would appear are varied and uncertain.  Over the large search regions
and over the span of days or weeks, many electromagnetic transients
typically occur, and with the wide variety of models it will be
difficult to associate unambigously a particular electromagnetic event with
a candidate gravitational-wave event.

The purpose of this letter is to present a strategy to alleviate both
of these issues; that is, to reduce both the search region and the
time required to plan and begin observations.  We follow the spirit of
\citet{2015arXiv150803608G} to develop a galaxy catalogue to guide the
observational plan.  However, our goal here is to develop a nearly
complete catalogue at the expense of having less accurate estimates of
the redshifts of the galaxies within the catalogue.  The accuracy of
the galaxy distances needs to be only as good as the distance estmates
of the gravitational-wave events.  Additionally we will outline a
straightforward and rapid technique to generate a nearly optimal
observing plan to follow up the events rapidly (i.e. within a few
seconds of the trigger). 

\section{Bayesian Approach to Follow-Up}

Because we will be interested in the rapid follow-up of candidate
gravitational-wave events, we will focussed on the rapid Bayesian
reconsruction outlined by \citet{2015arXiv150803634S}, BAYESTAR.  At
the most basic level, BAYESTAR yields a probability map on the sky in
the form of a HEALPpix map \citep{2005ApJ...622..759G} where each
pixel contains the probability $P(d|m)$ that a particular model
(i.e. position on the sky) will yield the data (i.e. the observed
strains on the LIGO and Virgo interferometers).  To plan an observing
strategy one would like the probability of a particular model
(i.e. position on the sky) given the data.  We have from Bayes's
theorem
\begin{equation}
  P(\mathrm{position}|\mathrm{data}) = \frac{P(\mathrm{position}
    P(\mathrm{data}|\mathrm{position})}{P(\mathrm{data})}.
  \label{eq:1}
\end{equation}
If we make the additional mild assumption that gravitational-waves
originate from nearly galaxies, the probability of a given position on
the sky naturally is proportional to the surface density of galaxies
in that direction within the distance range determined from the
modelling of the gravitational waveform.  Of course, these distance
estimates will usually have large uncertainities so the distance range
over which to integrate the galaxy density distribution will also be
large, so highly accurate redshift information is not needed to
construct $P(\mathrm{position})$.

Furthermore, because we will ultimately be interested in which fields
to observe (not which particular galaxies), accurate positions are not
required in the construction of $P(\mathrm{position})$. It is natural
to sample $P(\mathrm{position})$ also as a HEALPix grid with each pixel
covering about the same solid angle as the field of view of the
telescope of interest or the BAYESTAR map (a HEALPix level of 512 or
about 50 square arcminutes per pixel), so positions no more accurate
than arcminutes are required.  The key to generate the observing plan
rapidly is to calculate the required galaxy density maps beforehand in
principle at the desired resolution (this optimization only speeds the
process up slightly) for the distance ranges of interest.  With the
arrival of an alert, all that is required is to calculate
Eq.~(\ref{eq:1}) using the HEALPix maps, resample to the scale of the
telescope, renormalize the probability, sort the pixels from most
likely to least and output the positions to cover a given amount of
cumulative probability.

\section{Galaxy Catalogues}



\begin{figure}
  \includegraphics[width=\columnwidth,clip,trim=0 1.35cm 0 0]{T125738_bayestar_2MPZgz_0.03_0.04.pdf} 
  \includegraphics[width=\columnwidth,clip,trim=0 0 0 0.15cm]{T125738_bayestar_2MPZgz_0.01_0.05.pdf} 
  \caption{The number of fields required to cover the given fraction
    of the probablity region for a simulated LIGO detection (red
    curves without the galaxy map, green curves with a smoothed
    galaxy, blue curves with a raw galaxy map).  The upper solid
    curves use a healpix map with about 200,000 cells, the dashed
    curves have about 50,000 cells, the dotted curves have about
    12,000 cells and lower solid curves have about 3,000 cells,
    corresponding 0.2, 0.8, 3.2 and 13 square-degree fields of
    view. The redshift range of the galaxy map in the upper panel is
    $0.03<z<0.04$ and $0.01<z<0.05$ in the lowel panel.}
\end{figure}

This work was supported by the Natural Sciences and Engineering
Research Council of Canada, the Canadian Foundation for Innovation,
the British Columbia Knowledge Development Fund and the Bertha and
Louis Weinstein Research Fund at the University of British Columbia.

\bibliography{obsplan}
\bibliographystyle{apj}


\label{lastpage}
\end{document}
